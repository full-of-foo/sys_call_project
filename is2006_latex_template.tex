\documentclass{article}
\usepackage{interspeech2006,amssymb,amsmath,epsfig}
\usepackage{hyperref} % For hyperlinks in the PDF

\usepackage{natbib} % Harvard style bib
\bibliographystyle{IEEEtranN}

\usepackage{lipsum} % Dummy text

\setcounter{page}{1}
\sloppy		% better line breaks
\ninept
%SM below a registered trademark definition
\def\reg{{\rm\ooalign{\hfil
     \raise.07ex\hbox{\scriptsize R}\hfil\crcr\mathhexbox20D}}}


\title{System Software: Adding System Calls to the Linux Kernel}

\makeatletter
\def\name#1{\gdef\@name{#1\\}}
\makeatother
\name{{\em Anthony Troy and Liam Allen}}

\address{School of Computing  \\
Dublin City University \\
Dublin 9, Ireland\\
{\small \tt\{anthony.troy3,liam.allen5\}@mail.dcu.ie}
}

% You should submit a brief report, maximum of 8 pages, describing how you implemented the call, details of the changes made to the system files. 
% A listing of the application program, and a C program example using it.


\begin{document}

\maketitle

\begin{abstract}
Linux is arguably the most extensible and powerful operating system in existence today. 
Now with supports for ARM, x86-64, Alpha, PowerPC, SPARC and other architectures, 
the kernel is running on everything from small consumer electronics to supercomputers.
Mainstream Linux distributions aim to serve the needs of the majority, providing
interoperability with most common hardware such as display, sound and wireless devices.
Nevertheless, the exact needs of a user are rarely met by one distribution.
Accordingly one can customise the kernel for their given environment.
These adjustments might simply involve the tweaking of kernel parameters or removing
unneeded modules. More intricate alternations include adding custom 
kernel patches and modules. 
This papers briefly reviews the 
fundamental mechanism which facilitates this
extensibility, the kernel system call interface. Subsequently, in conjunction 
with our accompanying source code, we contribute the implementation of two trivial Linux system
calls.
\end{abstract}

\section{Introduction}
Since being first developed in 1991 by Linus Torvalds, Linux has quickly 
the surpassed the hobbyist market to becoming widely adopted in 
production environments. It currently powers everything from ``cell 
phones and embedded devices to more than 90 percent of the world's top 
500 supercomputers" \citep{Love7}. Then studying at the University of Helsinki,
Torvalds was abashed by the state of the operating systems (OS) space. Dominant 
OSs at this time were either proprietary or inextensible Unix systems. 
Accordingly, Torvalds released an immature but full-fledged Unix that was extensible
by design and licence. It is important to note that the initial success of the project 
would not have been possible without this permissive open-source license. 
Not long thereafter, Linux quickly attracted interests of many and ``evolved into a collaborative 
project" \citep{Love}.
\par
Today Torvalds continues to be the principal author and maintainer of the kernel while 
being backed by a sparse group of contributors. These developers are primarily 
sponsored by corporations such as Red Hat, Intel, IBM, Texas Instruments, Google
and Samsung, though anyone can potentially contribute. Being licensed under the
``GNU General Public License (GPL) version 2.0", one can download the kernel source
code and modify it, however distributing any changes requires making the altered source 
code available \citep{Love}. 




\section{Related work}
\lipsum[1]

\section{Study Design}
\lipsum[1]
\begin{itemize}
\item \lipsum[1]
\end{itemize}

\subsubsection{Todo - Some title}
\lipsum[1]
The kernel source code should also never be placed in the /usr/src/linux/ direc- tory, as that is the location of the kernel that the system libraries were built against,notyournewcustomkernel.Donotdoanykerneldevelopmentunder the /usr/src/ directory tree at all, but only in a local user directory where nothing bad can happen to the system.
(Nutshell)

\subsubsection{Todo - Another title}
\lipsum[1]


\section{Discussion}
\lipsum[1]

\section{Conclusions}
\lipsum[1]


\vspace{-7.5mm}
\renewcommand{\refname}{\section{References}}
\bibliography{is2006_latex_template}

\end{document}
