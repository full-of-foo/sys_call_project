\documentclass{article}
\usepackage{interspeech2006,amssymb,amsmath,epsfig}
\usepackage{hyperref} % For hyperlinks in the PDF

\usepackage{natbib} % Harvard style bib
\bibliographystyle{IEEEtranN}

\usepackage{lipsum} % Dummy text

\setcounter{page}{1}
\sloppy		% better line breaks
\ninept
%SM below a registered trademark definition
\def\reg{{\rm\ooalign{\hfil
     \raise.07ex\hbox{\scriptsize R}\hfil\crcr\mathhexbox20D}}}


\title{System Software: Adding System Calls to the Linux Kernel}

\makeatletter
\def\name#1{\gdef\@name{#1\\}}
\makeatother
\name{{\em Anthony Troy and Liam Allen}}

\address{School of Computing  \\
Dublin City University \\
Dublin 9, Ireland\\
{\small \tt\{anthony.troy3,liam.allen5\}@mail.dcu.ie}
}

% You should submit a brief report, maximum of 8 pages, describing how you implemented the call, details of the changes made to the system files. 
% A listing of the application program, and a C program example using it.


\begin{document}

\maketitle

\begin{abstract}
Linux is arguably the most extensible and powerful operating system in existence today. 
Now with supports for ARM, x86-64, Alpha, PowerPC, SPARC and other architectures, 
the kernel is running on everything from small consumer electronics to supercomputers.
Mainstream Linux distributions aim to serve the needs of the majority, providing
interoperability with most common hardware such as display, sound and wireless devices.
Nevertheless, the exact needs of a user are rarely met by one distribution.
Accordingly one can customise the kernel for their given environment.
These adjustments might simply involve the tweaking of kernel parameters or removing
unneeded modules. More intricate alternations include adding custom 
kernel patches and modules. 
This papers briefly reviews the 
fundamental mechanism which facilitates this
extensibility, the kernel system call interface. Subsequently, in conjunction 
with our accompanying source code, we contribute the implementation of two trivial Linux system
calls.





\end{abstract}

\section{Introduction}
Example citation \citep{db}. \lipsum[1]


\section{Related work}
\lipsum[1]

\section{Study Design}
\lipsum[1]
\begin{itemize}
\item \lipsum[1]
\end{itemize}

\subsubsection{Todo - Some title}
\lipsum[1]

\subsubsection{Todo - Another title}
\lipsum[1]


\section{Discussion}
\lipsum[1]

\section{Conclusions}
\lipsum[1]


\vspace{-7.5mm}
\renewcommand{\refname}{\section{References}}
\bibliography{is2006_latex_template}

\end{document}
